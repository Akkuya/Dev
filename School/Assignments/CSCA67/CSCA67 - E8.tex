\documentclass[]{article}

% Imports the catppuccin theme, using the mocha flavor,
% from the directory above. Actual implementation
% wouldn't need the import package unless the theme
% and the document are in different directories.
\usepackage{import}
\usepackage{xcolor}
% \usepackage{fancyhdr}
\usepackage{cancel}
\usepackage{mathtools}

% For permutations and combinations
\newcommand\Myperm[2][^n]{\prescript{#1\mkern-2.5mu}{}P_{#2}}
\newcommand\Mycomb[2]{\prescript{#1\mkern-0.5mu}{}C_{#2}}

% Colors
\definecolor{yorhabg}{HTML}{FFFFFF}
\definecolor{yorhafg}{HTML}{000000}
\definecolor{yorhagrid}{HTML}{B5AF9C}
\definecolor{mred}{HTML}{D67069}
\definecolor{mblue}{HTML}{6887A1}

\pagecolor{yorhabg}
\color{yorhafg}

\usepackage{preamble}

% Removes padding above title
\usepackage{titling}
\setlength{\droptitle}{-10em}

% Font package
\usepackage[T1]{fontenc}

\usepackage{fouriernc}

\usepackage{sectsty}
\usepackage{graphicx}
\usepackage{amsmath}
\usepackage{amsfonts}
\usepackage{amssymb}
\usepackage[skins, most]{tcolorbox}
\usepackage{enumitem}

\DeclareMathOperator{\sgn}{sgn}

\usepackage{tikz}
\usepackage{eso-pic}
\usetikzlibrary{calc,shadows.blur}
\usetikzlibrary{angles, quotes}
\usetikzlibrary{3d}

% Margins
\topmargin=0in
\evensidemargin=0in
\oddsidemargin=0in
\textwidth=6.5in
\textheight=9.0in
\headsep=0.25in

\AtBeginEnvironment{tcolorbox}{\small}

\newtcolorbox{imp}{enhanced,arc=0mm,colback=yorhabg,colframe=mred,leftrule=10mm,coltext=yorhafg,%
overlay={\node[anchor=west,outer sep=2pt] at (frame.west) {\includegraphics[width=6mm]{images/imageb.png}}; }}

\newtcolorbox{shortcut}{enhanced,arc=0mm,colback=yorhabg,colframe=mred,leftrule=10mm,coltext=yorhafg, coltitle=yorhabg, title=\texttt{Shortcut.}, 
overlay={\node[anchor=west,outer sep=2pt] at (frame.west) {\includegraphics[width=6mm]{images/imageb.png}}; }}

\newtcolorbox{question}[1]{
    enhanced, 
    colback=yorhabg,
    colframe=mblue,
    coltext=yorhafg,
    coltitle=yorhabg,
    attach boxed title to top left={yshift*=-\tcboxedtitleheight}, 
    title=\texttt{#1},
    boxed title size=title,
    boxed title style={%
        rounded corners=northeast, 
        rounded corners=northwest, 
        colback=tcbcolframe, 
        boxrule=0pt,
    },
    underlay boxed title={%
        \path[fill=tcbcolframe] (title.south west)--(title.south east) 
            to[out=0, in=180] ([xshift=5mm]title.east)--
            (title.center-|frame.east)
            [rounded corners=5pt] |- 
            (frame.north) -| cycle; 
    },
}
\newlength{\eqindent}
\settowidth{\eqindent}{$(n+1)^3 - (n+1)$}

\newcommand\bb[1]{\textcolor{yorhafg}{\textbf{#1}}}

\title{\textbf{CSCA67 - Exercises \#8}}
\author{Satyajit Datta \ 1012033336}
\date{\today}

\begin{document}

\maketitle
\begin{question}{2.1}
    For every integer $n \ge 1, n^3 - n$ is divisible by $3$.
\end{question}

Let $P(n)$ be $\exists z \in \mathbb{N}, n^3 - n = 3z$.

\underline{\textbf{Base Case}}

\begin{align*}
    & n^3 - n \\
    \Rightarrow & 1^3 - 1 \\
    = & 0 \\
    = & 3(0)
\end{align*}

Therefore, $P(1)$.
\medbreak
\underline{\textbf{Induction Step}}

\begin{flalign}
    &\text{Let k be arbitrary.} \\
    &\quad\text{Suppose } k \ge 1 \\
    &\quad\quad\text{Suppose } P(k) & \text{Induction Hypothesis}\\
    &\quad\quad\quad \exists z \in N, k^3 - k = 3z & \text{Defintion of } P\\
    &\quad\quad\quad k^3 - k = 3z & \text{4, E.I}\\
    &\quad\quad\quad (k+1)^3 - (k+1) = k^3 +3k^2 + 3k + 1 - k - 1 \\
    &\quad\quad\quad \hspace*{\eqindent} = (k^3 - k) + 3k^2 + 3k + 1 - 1 \nonumber \\
    &\quad\quad\quad \hspace*{\eqindent} = (k^3 - k) + 3k^2 + 3k \nonumber \\
    &\quad\quad\quad \hspace*{\eqindent} = 3z + 3k^2 + 3k & \text{IH}\nonumber \\
    &\quad\quad\quad \hspace*{\eqindent} = 3(z + k^2 + k) \nonumber \\
    &\quad\quad\quad (k+1)^3 - (k+1) = 3j & (z + k^2 + k) \in \mathbb{N} \\
    &\quad\quad\quad \exists z \in N, (k+1)^3 - (k+1) = 3z & \text{7, E.G}\\
    &\quad\quad\quad P(k+1) & \text{Definition of } P\\
    &\quad\quad P(k) \rightarrow P(k+1) & \text{3, 9, Implication}\\
    &\quad k \ge 1 \rightarrow (P(k) \rightarrow P(k+1)) & \text{2, 10, Implication}\\
    &\forall k, k \ge 1 \rightarrow (P(k) \rightarrow P(k+1)) & \text{1, 12, Implication}
\end{flalign}

Therefore, $\forall n \ge 1, P(n) \Longrightarrow \forall n \ge 1, \exists z \in \mathbb{N} \quad \text{s.t} \quad n^3 - n = 3z \blacksquare$.


\begin{question}{2.2}
    For every integer $n \ge 4, 2^n < n!$.
\end{question}

Let $P(n)$ be $2^n < n!$.

\underline{\textbf{Base Case}}

\begin{align*}
    & 2^n < n! \\
    \Rightarrow & 2^4 < 4! \\
    & 16 < 24 \\
    & T
\end{align*}

Therefore, $P(4)$.
\medbreak
\underline{\textbf{Induction Step}}
\newlength{\eqoindent}
\settowidth{\eqoindent}{$(2^k)(k)$}
\setcounter{equation}{0}
\begin{flalign}
    &\text{Let k be arbitrary.} \\
    &\quad\text{Suppose } k \ge 4 \\
    &\quad\quad\text{Suppose } P(k) & \text{Induction Hypothesis}\\
    &\quad\quad\quad 2^k < k! & \text{Defintion of } P\\
    &\quad\quad\quad (2^k)(k) < (k!)(2)\\
    &\quad\quad\quad 2^{k+1} < (k!)(2)\\
    &\quad\quad\quad \hspace*{\eqoindent} < (k!)(k+1) & (k \ge 4) \nonumber \\
    &\quad\quad\quad \hspace*{\eqoindent} < (k+1)! \nonumber \\
    &\quad\quad\quad P(k+1) & \text{Definition of } P\\
    &\quad\quad P(k) \rightarrow P(k+1) & \text{3, 7, Implication}\\
    &\quad k \ge 4 \rightarrow (P(k) \rightarrow P(k+1)) & \text{2, 8, Implication}\\
    &\forall k, k \ge 4 \rightarrow (P(k) \rightarrow P(k+1)) & \text{1, 9, Implication}
\end{flalign}

Therefore, $\forall n \ge 4, P(n) \Longrightarrow \forall n \ge 4, 2^n < n! \quad  \blacksquare$

\begin{question}{2.3}
    For every integer $n \ge 1$, if $a, r \in \mathbb{R}$  and $r \ne 1$, then $a+ar+ar^2+\ldots+ar^n = \frac{ar^n+1 - a}{r-1}$. Using summation notation,
    $\displaystyle \sum_{i = 0}^{n} ar^n = \frac{ar^{n+1} - a}{r-1}$
\end{question}

Let $P(n)$ be $\displaystyle \sum_{i = 0}^{n} ar^n = \frac{ar^{n+1} - a}{r-1}$

\underline{\textbf{Base Case}}

\begin{align*}
    \sum_{i = 0}^{n} ar^n = \frac{ar^n+1 - a}{r-1} \\
    \sum_{i = 0}^{1} ar^n = \frac{ar^2 - a}{r-1} \\
    a + ar = \frac{ar^2 - a}{r-1} \\
    a + ar = \frac{a(r^2 - 1)}{r-1} \\
    a + ar = \frac{a(r+1)(r-1)}{r-1} \\
    a + ar = a(r+1) \\
    a + ar = ar + a \\
    T
\end{align*}

Therefore, $P(1)$.
\medbreak
\underline{\textbf{Induction Step}}
\newlength{\equindent}
\settowidth{\equindent}{$\sum_{i = 0}^{n} ar^n + ar^{n+1}$}
\setcounter{equation}{0}
\begin{flalign}
    &\text{Let n be arbitrary.} \\
    &\quad\text{Suppose } n \ge 1 \\
    &\quad\quad\text{Suppose } P(n) & \text{Induction Hypothesis}\\
    &\quad\quad\quad \sum_{i = 0}^{n} ar^n = \frac{ar^{n+1} - a}{r-1} & \text{Defintion of } P\\
    &\quad\quad\quad \sum_{i = 0}^{n} ar^n + ar^{n+1} = \frac{ar^{n+1} - a}{r-1} + ar^{n+1}\\
    &\quad\quad\quad \hspace*{\eqoindent} = \frac{ar^{n+1} - a}{r-1} + \frac{(ar^{n+1})(r-1)}{r-1}\nonumber\\
    &\quad\quad\quad \hspace*{\eqoindent} = \frac{ar^{n+1} - a + ar^{n+1}(r-1)}{r-1}\nonumber\\
    &\quad\quad\quad \hspace*{\eqoindent} = \frac{ar^{n+1} - a + ar^{n+2}-ar^{n+1}}{r-1}\nonumber\\
    &\quad\quad\quad \hspace*{\eqoindent} = \frac{ar^{n+2} - a}{r-1}\nonumber\\
    &\quad\quad\quad \hspace*{\eqoindent} = \frac{ar^{(n+1) + 1} - a}{r-1}\nonumber\\
    &\quad\quad\quad \sum_{i = 0}^{n+1} ar^n = \frac{ar^{(n+1) + 1} - a}{r-1}\\
    &\quad\quad\quad P(n+1) & \text{Definition of } P\\
    &\quad\quad P(n) \rightarrow P(n+1) & \text{3, 7, Implication}\\
    &\quad n \ge 1 \rightarrow (P(n) \rightarrow P(n+1)) & \text{2, 8, Implication}\\
    &\forall n, n \ge 1 \rightarrow (P(n) \rightarrow P(n+1)) & \text{1, 9, Implication}
\end{flalign}

Therefore, $\displaystyle \forall n \ge 1, P(n) \Longrightarrow \forall n \ge 1, \sum_{i = 0}^{n} ar^n = \frac{ar^{n+1} - a}{r-1} \quad \blacksquare$.

\begin{question}{2.4}
    For every integer $\displaystyle n \ge 2, \sum_{i = 0}^{n} \frac{1}{i^2} < 2 - \frac{1}{n}$
\end{question}

Let $P(n)$ be $\displaystyle \sum_{i = 0}^{n} \frac{1}{i^2} < 2 - \frac{1}{n}$.

\underline{\textbf{Base Case}}

\begin{align*}
    & \sum_{i = 0}^{n} \frac{1}{i^2} < 2 - \frac{1}{n} \\
    & \sum_{i = 0}^{2} \frac{1}{i^2} < 2 - \frac{1}{2} \\
    & 1 + \frac{1}{2^2} < 2 - \frac{1}{2} \\
    & 1 + \frac{1}{4} < 2 - \frac{1}{2} \\
    & \frac{5}{4} < \frac{3}{2} \\
    & T
\end{align*}

Therefore, $P(2)$.
\medbreak
\underline{\textbf{Induction Step}}

\newlength{\eqqindent}
\settowidth{\eqqindent}{$\sum_{i = 0}^{n} \frac{1}{i^2} + \frac{1}{(n+1)^2}$}
\setcounter{equation}{0}
\begin{flalign}
    &\text{Let n be arbitrary.} \\
    &\quad\text{Suppose } n \ge 2 \\
    &\quad\quad\text{Suppose } P(n) & \text{Induction Hypothesis}\\
    &\quad\quad\quad \sum_{i = 0}^{n} \frac{1}{i^2} < 2 - \frac{1}{n} & \text{Defintion of } P\\
    &\quad\quad\quad \sum_{i = 0}^{n} \frac{1}{i^2} + \frac{1}{(n+1)^2} < 2  - \frac{1}{n} + \frac{1}{(n+1)^2}\\
    &\quad\quad\quad \hspace*{\eqqindent} = 2  - \frac{(n+1)^2 + n}{n(n+1)^2} \nonumber\\
    &\quad\quad\quad \hspace*{\eqqindent} < 2  - \frac{(n+1)}{n(n+1)^2} \nonumber & n \ge 2\\
    &\quad\quad\quad \hspace*{\eqqindent} = 2  - \frac{1}{n(n+1)} \nonumber\\
    &\quad\quad\quad \hspace*{\eqqindent} < 2  - \frac{1}{n+1}\nonumber & n \ge 2\\
    &\quad\quad\quad \sum_{i = 0}^{n+1} \frac{1}{i^2} < \frac{1}{n+1}\\
    &\quad\quad\quad P(n+1) & \text{Definition of } P\\
    &\quad\quad P(n) \rightarrow P(n+1) & \text{3, 7, Implication}\\
    &\quad n \ge 2 \rightarrow (P(n) \rightarrow P(n+1)) & \text{2, 8, Implication}\\
    &\forall n, n \ge 2 \rightarrow (P(n) \rightarrow P(n+1)) & \text{1, 9, Implication}
\end{flalign}

Therefore, $\forall n \ge 1, P(n) \Longrightarrow \forall n \ge 2, \sum_{i = 0}^{n} \frac{1}{i^2} + \frac{1}{(n+1)^2} \quad \blacksquare$.

\end{document}