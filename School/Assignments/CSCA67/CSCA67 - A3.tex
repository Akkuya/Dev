\documentclass[]{article}

% Imports the catppuccin theme, using the mocha flavor,
% from the directory above. Actual implementation
% wouldn't need the import package unless the theme
% and the document are in different directories.
\usepackage{import}
\usepackage{xcolor}
% \usepackage{fancyhdr}
\usepackage{cancel}
\usepackage{mathtools}

% For permutations and combinations
\newcommand\Myperm[2][^n]{\prescript{#1\mkern-2.5mu}{}P_{#2}}
\newcommand\Mycomb[2]{\prescript{#1\mkern-0.5mu}{}C_{#2}}

% Colors
\definecolor{yorhabg}{HTML}{FFFFFF}
\definecolor{yorhafg}{HTML}{000000}
\definecolor{yorhagrid}{HTML}{B5AF9C}
\definecolor{mred}{HTML}{D67069}
\definecolor{mblue}{HTML}{6887A1}

\pagecolor{yorhabg}
\color{yorhafg}

\usepackage{preamble}

% Removes padding above title
\usepackage{titling}
\setlength{\droptitle}{-10em}

% Font package
\usepackage[T1]{fontenc}

\usepackage{fouriernc}

\usepackage{sectsty}
\usepackage{graphicx}
\usepackage{amsmath}
\usepackage{amsfonts}
\usepackage{amssymb}
\usepackage[skins, most]{tcolorbox}
\usepackage{enumitem}

\DeclareMathOperator{\sgn}{sgn}

\usepackage{tikz}
\usepackage{eso-pic}
\usetikzlibrary{calc,shadows.blur}
\usetikzlibrary{angles, quotes}
\usetikzlibrary{3d}

% Margins
\topmargin=0in
\evensidemargin=0in
\oddsidemargin=0in
\textwidth=6.5in
\textheight=9.0in
\headsep=0.25in

\AtBeginEnvironment{tcolorbox}{\small}

\newtcolorbox{imp}{enhanced,arc=0mm,colback=yorhabg,colframe=mred,leftrule=10mm,coltext=yorhafg,%
overlay={\node[anchor=west,outer sep=2pt] at (frame.west) {\includegraphics[width=6mm]{images/imageb.png}}; }}

\newtcolorbox{shortcut}{enhanced,arc=0mm,colback=yorhabg,colframe=mred,leftrule=10mm,coltext=yorhafg, coltitle=yorhabg, title=\texttt{Shortcut.}, 
overlay={\node[anchor=west,outer sep=2pt] at (frame.west) {\includegraphics[width=6mm]{images/imageb.png}}; }}

\newtcolorbox{question}[1]{
    enhanced, 
    colback=yorhabg,
    colframe=mblue,
    coltext=yorhafg,
    coltitle=yorhabg,
    attach boxed title to top left={yshift*=-\tcboxedtitleheight}, 
    title=\texttt{#1},
    boxed title size=title,
    boxed title style={%
        rounded corners=northeast, 
        rounded corners=northwest, 
        colback=tcbcolframe, 
        boxrule=0pt,
    },
    underlay boxed title={%
        \path[fill=tcbcolframe] (title.south west)--(title.south east) 
            to[out=0, in=180] ([xshift=5mm]title.east)--
            (title.center-|frame.east)
            [rounded corners=5pt] |- 
            (frame.north) -| cycle; 
    },
}

\newcommand\bb[1]{\textcolor{yorhafg}{\textbf{#1}}}

\title{\textbf{CSCA67 - Assignment \#3}}
\author{Satyajit Datta \ 1012033336}
\date{\today}

\begin{document}

\maketitle
\section*{1. Proof Strategies}
\begin{question}{1.1}
    We write a|b, read “a divides b”, to stand for $\exists k, b = ak$. Prove that if 4 divides $a-b$, then it also divides $a^2 - b^2$.
\end{question}

\underline{\textbf{Proof}}
\newlength{\eqaindent}
\settowidth{\eqaindent}{$a^2-b^2$}

\begin{flalign}
    &\text{Suppose } 4\mid(a+b)\\
    &\quad \exists k, a+b = 4k &\text{1, Def of $x\mid y$} \\
    &\quad a + b = 4k &\text{2, E.I} \\
    &\quad a^2 - b^2 = (a+b)(a-b) \\
    &\quad\hspace*{\eqaindent} = 4k(a-b) &\text{Math} \nonumber\\
    &\quad\exists k, a^2 - b^2 = 4k &\text{4, E.G}\\
    &\quad 4\mid(a^2-b^2) &\text{5, Def of $x \mid y$}\\
    &4\mid(a+b) \rightarrow 4\mid(a^2 + b^2) &\text{1, 6, Implication}
\end{flalign}

As required to prove. $\blacksquare$.

\begin{question}{1.2}
    For all real numbers $x$ and $y$, $\min(x, y) = (x + y - \mid x - y\mid)/2$ and $\max(x, y) =  (x + y + \mid x - y\mid)/2$
\end{question}

WTS: $\forall x, y, \min(x, y) = (x + y - \mid x - y\mid)/2$

\newlength{\spacingT}
\settowidth{\spacingT}{$(x + y - \mid x - y\mid)/2$}

\setcounter{equation}{0}
\begin{flalign}
    &\text{Let x, y be arbitrary}. \\
    &\quad \text{Assume } x\le y \\
    &\quad\quad \min(x, y) = x &\text{2}\\
    &\quad\quad x - y \le 0 &\text{2} \\
    &\quad\quad \mid x - y \mid = -(x - y) &\text{4, def$^n$ of $\mid \cdot\mid$} \\
    &\quad\quad (x + y - \mid x - y\mid)/2 = (x + y - (-(x - y)))/2 &\text{5} \\
    &\quad\quad \hspace*{\spacingT} = (x - y - (-x + y))/2 \nonumber \\
    &\quad\quad \hspace*{\spacingT} = 2x/2 \nonumber \\
    &\quad\quad \hspace*{\spacingT} = x \nonumber \\
    &\quad\quad \min(x, y) = (x + y - \mid x - y\mid)/2 &\text{3, 6} \\
    &\quad x \le y \rightarrow (\min(x, y) = (x + y - \mid x - y\mid)/2) &\text{2, 7, implication}\\
    &\quad \text{Assume } x \ge y \\
    &\quad\quad \min(x, y) = y &\text{9}\\
    &\quad\quad x - y \ge 0 &\text{9} \\
    &\quad\quad \mid x - y \mid = x - y &\text{11, def$^n$ of $\mid \cdot\mid$} \\
    &\quad\quad (x + y - \mid x - y\mid)/2 = (x + y - (x - y))/2 &\text{12} \\
    &\quad\quad \hspace*{\spacingT} = (x - y -x + y)/2 \nonumber \\
    &\quad\quad \hspace*{\spacingT} = 2y/2 \nonumber \\
    &\quad\quad \hspace*{\spacingT} = y \nonumber \\
    &\quad\quad \min(x, y) = (x + y - \mid x - y\mid)/2 &\text{10, 13} \\
    &\quad x \ge y \rightarrow (\min(x, y) = (x + y - \mid x - y\mid)/2) &\text{9, 14, implication}\\
    &\quad\min(x, y) = (x + y - \mid x - y\mid)/2 &\text{8, 15, Cases} \\
    &\forall x, y, \min(x, y) = (x + y - \mid x - y\mid)/2 &\text{}{1, 16, U.G}
\end{flalign}

As required to show. $\blacksquare$.
\medbreak
WTS: $\max(x, y) =  (x + y + \mid x - y\mid)/2$

\newlength{\spacingG}
\settowidth{\spacingT}{$(x + y + \mid x - y\mid)/2$}

\setcounter{equation}{0}
\begin{flalign}
    &\text{Let x, y be arbitrary}. \\
    &\quad \text{Assume } x\le y \\
    &\quad\quad \max(x, y) = y &\text{2}\\
    &\quad\quad x - y \le 0 &\text{2} \\
    &\quad\quad \mid x - y \mid = -(x - y) &\text{4, def$^n$ of $\mid \cdot\mid$} \\
    &\quad\quad (x + y + \mid x - y\mid)/2 = (x + y + (-(x - y)))/2 &\text{5} \\
    &\quad\quad \hspace*{\spacingT} = (x - y + (-x + y))/2 \nonumber \\
    &\quad\quad \hspace*{\spacingT} = 2y/2 \nonumber \\
    &\quad\quad \hspace*{\spacingT} = y \nonumber \\
    &\quad\quad \max(x, y) = (x + y + \mid x - y\mid)/2 &\text{3, 6} \\
    &\quad x \le y \rightarrow (\max(x, y) = (x + y + \mid x - y\mid)/2) &\text{2, 7, implication}\\
    &\quad \text{Assume } x \ge y \\
    &\quad\quad \max(x, y) = x &\text{9}\\
    &\quad\quad x - y \ge 0 &\text{9} \\
    &\quad\quad \mid x - y \mid = x - y &\text{11, def$^n$ of $\mid \cdot\mid$} \\
    &\quad\quad (x + y + \mid x - y\mid)/2 = (x + y + (x - y))/2 &\text{12} \\
    &\quad\quad \hspace*{\spacingT} = 2x/2 \nonumber \\
    &\quad\quad \hspace*{\spacingT} = x \nonumber \\
    &\quad\quad \max(x, y) = (x + y + \mid x - y\mid)/2 &\text{10, 13} \\
    &\quad x \ge y \rightarrow (\max(x, y) = (x + y + \mid x - y\mid)/2) &\text{9, 14, implication}\\
    &\quad\max(x, y) = (x + y + \mid x - y\mid)/2 &\text{8, 15, Cases} \\
    &\forall x, y, \max(x, y) = (x + y + \mid x - y\mid)/2 &\text{}{1, 16, U.G}
\end{flalign}

As required to show. $\blacksquare$.

\begin{question}{1.3}
    Recall that x is irrational if there are no integers a and b, such that $x = \frac{a}{b}$. Prove that the third power of any
real number is irrational only if the number itself is irrational.
\end{question}

WTS: $x^3 \notin \mathbb{Q} \rightarrow x \notin \mathbb{Q}$

\underline{\textbf{Proof.}}

\setcounter{equation}{0}
\begin{flalign}
    &\text{Suppose } x \in \mathbb{Q} \\
    &\quad \exists a, b \in \mathbb{Z} (x = \frac{a}{b}) \land (b \ne 0) &\text{1, Def$^n$ of $\mathbb{Q}$} \\
    &\quad (x = \frac{a}{b}) \land (b \ne 0) &\text{2, E.I} \\
    &\quad x = \frac{a}{b} &\text{3, simp.} \\
    &\quad b \ne 0 &\text{3, simp.} \\
    &\quad x^3 = \frac{a^3}{b^3} &\text{4, math} \\
    &\quad c = a^3 &\text{2, 6} \\
    &\quad d = b^3 &\text{2, 6} \\
    &\quad d \ne 0 &\text{7, 5} \\
    &\quad x^3 = \frac{c}{d} &\text{6, 7, 8}\\
    &\quad (x^3 = \frac{c}{d}) \land (d \ne 0) &\text{9, 10, conj.}\\
    &\quad \exists a, b \in \mathbb{Z}, (x^3 = \frac{a}{b}) \land (b \ne 0) &\text{11, E.G}\\
    &\quad x^3 \in \mathbb{Q} &\text{12, Def$^n$ of $\mathbb{Q}$}\\
    & x \in \mathbb{Q} \rightarrow x^3 \in \mathbb{Q} &\text{1, 13, imp.} \\
    & x^3 \notin \mathbb{Q} \rightarrow x \notin \mathbb{Q} &\text{14, contr.}
\end{flalign}

As required to show. $\blacksquare$.

\begin{question}{1.4}
    Prove that if x is rational, then $x + \sqrt{2}$ is not. You may use the fact that $\sqrt{2}$ is irrational, as we proved in class.
\end{question}
\newlength{\spacingA}
\settowidth{\spacingA}{$x + \sqrt{x} - x$}
\setcounter{equation}{0}
\begin{flalign}
    &\text{Suppose } x \in \mathbb{Q} \\
    &\quad\text{Suppose } x + \sqrt{2} \in \mathbb{Q} &\text{For contradiction} \\
    &\quad\quad \exists a, b \in \mathbb{Z}, (x = \frac{a}{b}) \land (b \ne 0) &\text{1, Def$^n$ of $\mathbb{Q}$} \\
    &\quad\quad \exists a, b \in \mathbb{Z}, (x + \sqrt{2} = \frac{a}{b}) \land (b \ne 0) &\text{2, Def$^n$ of $\mathbb{Q}$} \\
    &\quad\quad (x = \frac{a}{b}) \land (b \ne 0) &\text{3, E.I} \\
    &\quad\quad (x + \sqrt{2} = \frac{c}{d}) \land (d \ne 0) &\text{4, E.I} \\
    &\quad\quad x = \frac{a}{b} &\text{5, simp.} \\
    &\quad\quad x + \sqrt{2} = \frac{c}{d} &\text{6, simp.} \\
    &\quad\quad b \ne 0 &\text{5, simp} \\
    &\quad\quad d \ne 0 &\text{6, simp} \\
    &\quad\quad x + \sqrt{2} - x = \frac{a}{b} - \frac{c}{d} &\text{7, 8} \\
    &\quad\quad \hspace*{\spacingA} = \frac{ad - bc}{bd} \nonumber \\
    &\quad\quad \sqrt{2} = \frac{ad - bc}{bd} &\text{11, math} \\
    &\quad\quad bd \ne 0 &\text{9, 10} \\
    &\quad\quad (\sqrt{2} = \frac{ad - bc}{bd}) \land (bd \ne 0) &\text{12, 13, conj.} \\
    &\quad\quad \exists f, g \in \mathbb{Z}, (\sqrt{x} = \frac{f}{g}) \land (g \ne 0) &\text{14, E.G} \\
    &\quad\quad \sqrt{2} \in \mathbb{Q} &\text{15, Def$^n$ of $\mathbb{Q}$} \\
    &\quad\quad \text{A contradiction} \\
    &\quad x + \sqrt{2} \notin \mathbb{Q} &\text{2, 16, contradiction} \\
    &x \in \mathbb{Q} \rightarrow x + \sqrt{2} \notin \mathbb{Q} &\text{1, 17, imp.} 
\end{flalign}

As required to show. $\blacksquare$.

\begin{question}{1.5}
    Using the same definition of rational as above, prove or disprove: If x is irrational, then so is $x + \sqrt{2}$
\end{question}

We will disprove this statement.

Let $x = -\sqrt{2}$

Therefore, x is irrational.

$x + \sqrt{2} = -\sqrt{2} + \sqrt{2} = 0$

0 is rational.

Therefore, we have shown a counterexample where the hypothesis is true and the conclusion is false, therefore this statement is not valid.

\section*{2. Induction}

\begin{question}{2.1}
    Prove that: 
    \begin{align*}
        \sum_{i = 1}^{n} i^2 = \frac{n(n+1)(2n+1)}{6}
    \end{align*}

    for all positive integers $n$.
\end{question}

\underline{\textbf{Base Case.}}
\begin{align*}
    \sum_{i = 1}^{n} i^2 = 1 \\
    \frac{(1)(1+1)(2+1)}{6} = 1
\end{align*}

Therefore, P(1)

Let $P(n)$ be $\sum_{i = 1}^{n} i^2 = \frac{n(n+1)(2n+1)}{6}$.

WTS: $\forall k \ge 0, P(k) \rightarrow P(k+1)$

\newlength{\spacingB}
\settowidth{\spacingB}{$\sum_{i = 1}^{n+1} i^2$}
\setcounter{equation}{0}
\begin{flalign}
    &\text{Let k be arbitrary} \\
    &\quad\text{Suppose } k \ge 1 \\
    &\quad\quad\text{Suppose P(k)} &\text{Induction Hypothesis} \\
    &\quad\quad\quad \sum_{i = 1}^{k} i^2 = \frac{k(k+1)(2k+1)}{6} &\text{3, Def$^n$ of P} \\
    &\quad\quad\quad \sum_{i = 1}^{k+1} i^2 = \sum_{i = 1}^{k} i^2 + (k+1)^2 &\text{math} \\
    &\quad\quad\quad \hspace*{\spacingB} = \frac{k(k+1)(2k+1)}{6} + (n+1)^2 &\text{IH} \nonumber\\
    &\quad\quad\quad \hspace*{\spacingB} = \frac{k(k+1)(2k+1)}{6} + \nonumber \\
    &\quad\quad\quad \hspace*{\spacingB} = \frac{k(k+1)(2k+1) + 6(k+1)^2}{6} \nonumber \\
    &\quad\quad\quad \hspace*{\spacingB} = \frac{(k+1)(k(2k+1) + 6(k+1))}{6} \nonumber \\
    &\quad\quad\quad \hspace*{\spacingB} = \frac{(k+1)(2k^2+k + 6k+6)}{6} \nonumber \\ 
    &\quad\quad\quad \hspace*{\spacingB} = \frac{(k+1)(2k^2+7k+6)}{6} \nonumber\\ 
    &\quad\quad\quad \hspace*{\spacingB} = \frac{(k+1)(k+3)(2k+3)}{6} \nonumber \\
    &\quad\quad\quad \hspace*{\spacingB} = \frac{(k+1)(k+2)(2k+3)}{6} \nonumber \\
    &\quad\quad\quad \hspace*{\spacingB} = \frac{(k+1)((k+1)+1)(2(k+1)+1)}{6} \nonumber \\
    &\quad\quad\quad P(k+1) &\text{5, Def$^n$ of P}\\
    &\quad\quad P(k) \rightarrow P(k+1) &\text{3, 6, implication}\\
    &\quad k \ge 1 \rightarrow (P(k) \rightarrow P(k+1)) &\text{2, 7, implication}\\
    &\forall k \ge 1, P(k) \rightarrow P(k+1) &\text{1, 8, U.G}
\end{flalign}

Therefore, $\forall n \ge 1, P(n)$, as required to show. $\blacksquare$.

\begin{question}{2.2}
    Prove that
    \begin{align*}
        n! > 2^n 
    \end{align*}

    for all $n \ge 4$
\end{question}

\underline{\textbf{Base Case.}}
\begin{align*}
    4! = 24 \\
    2^4 = 16
\end{align*}

Therefore, P(4)

Let $P(n)$ be $n! > 2^n$.

WTS: $\forall k \ge 4, P(k) \rightarrow P(k+1)$

\newlength{\spacingC}
\settowidth{\spacingC}{$(k+1)!$}
\setcounter{equation}{0}
\begin{flalign}
    &\text{Let k be arbitrary} \\
    &\quad\text{Suppose } k \ge 4 \\
    &\quad\quad\text{Suppose P(k)} &\text{Induction Hypothesis} \\
    &\quad\quad\quad (k+1)! = k! \cdot (k+1) &\text{Math} \\
    &\quad\quad\quad \hspace*{\spacingC} < 2^n \cdot (k+1) &\text{IH} \nonumber\\
    &\quad\quad\quad \hspace*{\spacingC} < 2^n \cdot 2 &\text{$2 \le 4 \le k$} \nonumber\\
    &\quad\quad\quad \hspace*{\spacingC} = 2^{n+1} \nonumber\\
    &\quad\quad\quad P(k+1) &\text{4, Def$^n$ of P}\\
    &\quad\quad P(k) \rightarrow P(k+1) &\text{3, 5, implication}\\
    &\quad k \ge 4 \rightarrow (P(k) \rightarrow P(k+1)) &\text{2, 6, implication}\\
    &\forall k \ge 4, P(k) \rightarrow P(k+1) &\text{1, 7, U.G}
\end{flalign}

Therefore, $\forall n \ge 4, P(n)$, as required to show. $\blacksquare$.

\begin{question}{2.3}
    Prove that $3^{2n} - 1$ is divisible by $8$, for all positive integers $n$.
\end{question}

\underline{\textbf{Base Case.}}
\begin{align*}
    3^{2} - 1 = 8 \\
    8 \mid 8
\end{align*}

Therefore, P(1)

Let $P(n)$ be $\exists k, 3^{2n} - 1 = 8k$.

WTS: $\forall k \ge 0, P(k) \rightarrow P(k+1)$

\newlength{\spacingD}
\settowidth{\spacingD}{$3^{2(k+1)} - 1$}
\setcounter{equation}{0}
\begin{flalign}
    &\text{Let k be arbitrary} \\
    &\quad\text{Suppose } k \ge 1 \\
    &\quad\quad\text{Suppose P(k)} &\text{Induction Hypothesis} \\
    &\quad\quad\quad \exists m, 3^{2k} -1 = 8m &\text{3, Def$^n$ of P} \\
    &\quad\quad\quad 3^{2k} - 1 = 8m &\text{4, E.I} \\
    &\quad\quad\quad 3^{2(k+1)} - 1 = 3^{2k+2} - 1 &\text{math} \\
    &\quad\quad\quad \hspace*{\spacingD} = (9)3^{2k} - 1 \nonumber \\
    &\quad\quad\quad \hspace*{\spacingD} = (8+1)3^{2k} - 1 \nonumber \\
    &\quad\quad\quad \hspace*{\spacingD} = 8(3^{2k}) + 3^{2k} - 1 \nonumber \\
    &\quad\quad\quad \hspace*{\spacingD} = 8(3^{2k}) + 3^{2k} - 1 \nonumber \\  
    &\quad\quad\quad \hspace*{\spacingD} = 8(3^{2k}) + 8m &\text{5, IH}\nonumber \\  
    &\quad\quad\quad \hspace*{\spacingD} = 8(3^{2k} + m) \nonumber \\  
    &\quad\quad\quad \exists m, 3^{2(k+1)} - 1 = 8m &\text{6, E.G} \\  
    &\quad\quad\quad P(k+1) &\text{7, Def$^n$ of P}\\  
    &\quad\quad P(k) \rightarrow P(k+1) &\text{3, 8, implication}\\
    &\quad k \ge 1 \rightarrow (P(k) \rightarrow P(k+1)) &\text{2, 9, implication}\\
    &\forall k \ge 1, P(k) \rightarrow P(k+1) &\text{1, 10, U.G}
\end{flalign}

Therefore, $\forall n \ge 1, P(n)$, as required to show. $\blacksquare$.

\begin{question}{2.4}
    Prove that
    \begin{align*}
        \sum_{j = 1}^{n} \frac{j}{(j+1)!} \le 1 - \frac{1}{(n+1)!}
    \end{align*}
    for all $n \ge 1$.
\end{question}

\underline{\textbf{Base Case.}}
\begin{align*}
    \sum_{j = 1}^{1} \frac{j}{(j+1)!} = \frac{1}{2} \\
    1 - \frac{1}{(1+1)!} = \frac{1}{2}
\end{align*}

Therefore, P(1)

Let $P(n)$ be $\sum_{j = 1}^{n} \frac{j}{(j+1)!} \le 1 - \frac{1}{(n+1)!}$.

WTS: $\forall k \ge 0, P(k) \rightarrow P(k+1)$

\newlength{\spacingE}
\settowidth{\spacingE}{$\sum_{j = 1}^{n} \frac{j}{(j+1)!}$}
\setcounter{equation}{0}
\begin{flalign}
    &\text{Let k be arbitrary} \\
    &\quad\text{Suppose } k \ge 1 \\
    &\quad\quad\text{Suppose P(k)} &\text{Induction Hypothesis} \\
    &\quad\quad\quad \sum_{j = 1}^{k+1} \frac{j}{(j+1)!} = \sum_{j = 1}^{k} \frac{j}{(j+1)!} + \frac{k+1}{(k+2)!} \\ 
    &\quad\quad\quad \hspace*{\spacingE} \le 1 - \frac{1}{(k+1)!} + \frac{k+1}{(k+2)!} &\text{IH} \nonumber \\
    &\quad\quad\quad \hspace*{\spacingE} = 1 - (\frac{1}{(k+1)!} - \frac{k+1}{(k+2)!}) \nonumber \\
    &\quad\quad\quad \hspace*{\spacingE} = 1 - (\frac{k+2}{(k+1)!(k+2)} - \frac{k+1}{(k+2)!})  \nonumber \\
    &\quad\quad\quad \hspace*{\spacingE} = 1 - (\frac{k+2}{(k+2)!} - \frac{k+1}{(k+2)!})  \nonumber \\
    &\quad\quad\quad \hspace*{\spacingE} = 1 - \frac{k+2 - (k+1)}{(k+2)!} \nonumber \\
    &\quad\quad\quad \hspace*{\spacingE} = 1 - \frac{1}{(k+2)!}  \nonumber \\
    &\quad\quad\quad \hspace*{\spacingE} = 1 - \frac{1}{((k+1)+1)!}  \nonumber \\
    &\quad\quad\quad P(k+1) &\text{4, Def$^n$ of P}\\  
    &\quad\quad P(k) \rightarrow P(k+1) &\text{3, 5, implication}\\
    &\quad k \ge 1 \rightarrow (P(k) \rightarrow P(k+1)) &\text{2, 6, implication}\\
    &\forall k \ge 1, P(k) \rightarrow P(k+1) &\text{1, 7, U.G}
\end{flalign}

Therefore, $\forall n \ge 1, P(n)$, as required to show. $\blacksquare$.

\begin{question}{3.5}
    Prove that if $A_1, A_2, \ldots A_n$ and $B$ are sets, then
    \begin{align*}
        (A_1 \ B) \cup (A_2 \ B)\cup \ldots \cup (A_n \ B) = (A_1 \cup A_2 \cup \ldots \cup A_n) \ B
    \end{align*}
    for all $n \ge 2$
\end{question}

\underline{\textbf{Base Case.}}
\begin{align*}
    (A_1 \ B) \cup (A_2 \ B)  &= {x : (x \in A_1 \land x \notin B) \lor (x \in A_2 \land x \notin B)} \\
    &= {x : (x \notin B) \land (x \in A_1 \lor x \in A_2)} \\
    &= (A_1 \cup A_2) \ B
\end{align*}

Therefore, P(2)

Let $P(n)$ be $(A_1 \ B) \cup (A_2 \ B)\cup \ldots \cup (A_n \ B) = (A_1 \cup A_2 \cup \ldots \cup A_n) \ B$.

WTS: $\forall k \ge 2, P(k) \rightarrow P(k+1)$

\newlength{\spacingF}
\settowidth{\spacingF}{$(A_1 \ B) \cup (A_2 \ B)\cup \ldots \cup (A_{n+1} \ B) \ B$}
\setcounter{equation}{0}
\begin{flalign}
    &\text{Let k be arbitrary} \\
    &\quad\text{Suppose } k \ge 2 \\
    &\quad\quad\text{Suppose P(k)} &\text{Induction Hypothesis} \\
    &\quad\quad\quad (A_1 \ B) \cup (A_2 \ B)\cup \ldots \cup (A_{k+1} \ B) = (A_1 \ B) \cup (A_2 \ B)\cup \ldots \cup (A_k \ B) \cup (A_{k+1} \ B) \\ 
    &\quad\quad\quad \hspace*{\spacingF} = (A_1 \cup A_2 \cup \ldots \cup A_k) \cup (A_{k+1} \ B) &\text{IH} \nonumber \\
    &\quad\quad\quad  = {x : (x \notin B \land (x \in A_1 \lor x \in A_2 \lor \ldots \lor A_k)) \lor (x \notin B \land x \in A_{k+1})} \nonumber \\
    &\quad\quad\quad \hspace*{\spacingF} = {x : (x \notin B) \land (x \in A_1 \lor x \in A_2 \lor \ldots  \lor x \in A_{k+1})} \nonumber \\
    &\quad\quad\quad \hspace*{\spacingF} = (A_1 \cup A_2 \cup \ldots\cup A_{k+1}) \ B \nonumber \\
    &\quad\quad\quad P(k+1) &\text{4, Def$^n$ of P}\\  
    &\quad\quad P(k) \rightarrow P(k+1) &\text{3, 5, implication}\\
    &\quad k \ge 2 \rightarrow (P(k) \rightarrow P(k+1)) &\text{2, 6, implication}\\
    &\forall k \ge 2, P(k) \rightarrow P(k+1) &\text{1, 7, U.G}
\end{flalign}

\section*{3. Quantifier and Logical Implication and Equivalence}
\begin{question}{3.1}
    \begin{align*}
        (\forall x, P(x)) \rightarrow (\exists x, Q(x)) \text{ and } \exists x, P(x) \rightarrow Q(x)
    \end{align*}
\end{question}

\setcounter{equation}{0}
\begin{flalign}
    &(\forall x, P(x)) \rightarrow (\exists x, Q(x)) \\
    &\neg(\forall x, P(x)) \lor (\exists x, Q(x)) &\text{cond.} \\
    &\exists x, \neg P(x) \lor \exists x, Q(x) &\text{q.neg} \\
    &\exists x, (\neg P(x) \lor Q(x)) &\text{distr.} \\
    &\exists x, (P(x) \rightarrow Q(x)) &\text{cond.}
\end{flalign}

Therefore, $A \rightarrow B$ and $B \rightarrow A$ are both tautologies, since the statements are logically equivalent.

\begin{question}{3.2}
    \begin{align*}
        \exists x\forall y, P(x, y) \text{ and } \forall y \exists x, P(x, y)
    \end{align*}
\end{question}

\setcounter{equation}{0}
\begin{flalign}
    &\exists x \forall y, P(x, y) \\
    &\forall y, P(a, y) &\text{1, E.I}\\
    &\text{Let y be arbitrary} \\
    &\quad P(a, y) &\text{2, 3, U.I} \\
    &\quad \exists x, P(x, y) &\text{4, E.G} \\
    &\forall y, \exists x, P(x, y) &\text{3, 5, U.G}
\end{flalign}

Thus, $A \rightarrow B$ is a tautology.

However, let $W \subseteq U = \emptyset$

For A, since there are no elements in W, the statement can never be true.

For B, since there are no elements in W, the statement is vacuously true.

Thus, we have shown a statement where $B \rightarrow A$ is not a tautology.


\begin{question}{3.3}
    Assuming a non-empty universe of discourse, $\exists x\forall y, P(x) \rightarrow Q(y)$ and $(\forall y, Q(y)) \lor (\exists x, \neg P(x))$.
\end{question}

\setcounter{equation}{0}
\begin{flalign}
    &\exists x\forall y, P(x) \rightarrow Q(y) \\
    &\forall y, P(a) \rightarrow Q(y) &\text{1, E.I} \\
    &\text{Let y be arbitrary} \\
    &\quad P(a) \rightarrow Q(y) &\text{2, 3, U.I} \\
    &\quad \neg P(a) \lor Q(y) &\text{4, cond.} \\
    &(\forall y, Q(y)) \lor \neg P(a) &\text{5, 3, U.G} \\
    &(\forall y, Q(y)) \lor (\exists x, \neg P(x)) &\text{7, E.G}
\end{flalign}

Therefore, $A \rightarrow B$ is a tautology.

\setcounter{equation}{0}
\begin{flalign}
    &(\forall y, Q(y)) \lor (\exists x, \neg P(x))
    &(\forall y, Q(y)) \lor \neg P(a) &\text{1, E.I} \\
    &\text{Let y be arbitrary} \\
    &\quad Q(y) \lor \neg P(a) &\text{2, 3 U.I} \\
    &\quad P(a) \rightarrow Q(y) &\text{4, cond.} \\
    &\forall y, P(a) \rightarrow Q(y) &\text{3, 5, U.G} \\
    &\exists x, \forall y, P(x) \rightarrow Q(y)
\end{flalign}

Therefore, $B \rightarrow A$ is a tautology.

\begin{question}{3.4}
    Assuming a non-empty universe of discourse, $\exists x\forall y, P(y) \rightarrow Q(x)$ and $(\forall y, P(y)) \rightarrow (\exists x, Q(x))$.
\end{question}

\setcounter{equation}{0}
\begin{flalign}
    &\text{Assume } \exists x\forall y, P(y) \rightarrow Q(x) \\
    &\quad \forall y, P(y) \rightarrow Q(a) &\text{1, E.I} \\
    &\quad \text{Let y be arbitrary} \\
    &\quad\quad \text{Assume P(y)} \\
    &\quad\quad\quad Q(a) &\text{4, 2, M.P}\\
    &\quad\quad \exists x Q(x) &\text{5, E.G}\\
    &\quad\quad P(y) \rightarrow \exists x, Q(x) &\text{4, 6, implication} \\
    &\quad (\forall y, P(y)) \rightarrow (\exists x, Q(x)) &\text{2, 7, U.G}  \\
    &\exists x\forall y, P(y) \rightarrow Q(x) \rightarrow (\forall y, Q(y)) \rightarrow (\exists x, P(x))
\end{flalign}

Thus, $A \rightarrow B$ is a tautology.

Let $W \subseteq U = {a, b}$, where $P(a) = T, Q(b) = False, P(b) = False, Q(b) = False$

A:

\begin{align*}
    \exists x\forall y, P(y) \rightarrow Q(x) \\
    ((P(a) \rightarrow Q(a)) \lor (P(a) \rightarrow Q(b))) \land ((P(b) \rightarrow Q(a)) \lor (P(b) \rightarrow Q(b))) \\
    ((T \rightarrow F) \lor (T \rightarrow F)) \land ((F \rightarrow F) \lor (F \rightarrow F)) \\
    ((F \lor  F) \land (T \lor T)) \\
    F \land T \\
    F
\end{align*}

\begin{align*}
    (\forall y, P(y)) \rightarrow (\exists x, Q(x)) \\
    (P(a) \land P(b)) \rightarrow (Q(a) \lor Q(b)) \\
    (T \land F) \rightarrow (F \lor F) \\
    F \rightarrow F \\
    T
\end{align*}

Thus, $B \rightarrow A$ is not a tautology.

$\exists a, b, (n = ab) \land (0 < a < n) \land (0 < b < n)$


\setcounter{equation}{0}
\begin{flalign}
    $\text{} \
\end{flalign}

\end{document}



