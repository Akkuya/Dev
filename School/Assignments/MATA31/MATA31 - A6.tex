\documentclass[]{article}

% Imports the catppuccin theme, using the mocha flavor,
% from the directory above. Actual implementation
% wouldn't need the import package unless the theme
% and the document are in different directories.
\usepackage{import}
\usepackage{xcolor}
% \usepackage{fancyhdr}
\usepackage{cancel}
\usepackage{mathtools}

% For permutations and combinations
\newcommand\Myperm[2][^n]{\prescript{#1\mkern-2.5mu}{}P_{#2}}
\newcommand\Mycomb[2]{\prescript{#1\mkern-0.5mu}{}C_{#2}}

% Colors
\definecolor{yorhabg}{HTML}{FFFFFF}
\definecolor{yorhafg}{HTML}{000000}
\definecolor{yorhagrid}{HTML}{B5AF9C}
\definecolor{mred}{HTML}{D67069}
\definecolor{mblue}{HTML}{6887A1}

\pagecolor{yorhabg}
\color{yorhafg}

\usepackage{preamble}

% Removes padding above title
\usepackage{titling}
\setlength{\droptitle}{-10em}

% Font package
\usepackage[T1]{fontenc}

\usepackage{fouriernc}

\usepackage{sectsty}
\usepackage{graphicx}
\usepackage{amsmath}
\usepackage{amsfonts}
\usepackage{amssymb}
\usepackage[skins, most]{tcolorbox}
\usepackage{enumitem}

\DeclareMathOperator{\sgn}{sgn}

\usepackage{tikz}
\usepackage{eso-pic}
\usetikzlibrary{calc,shadows.blur}
\usetikzlibrary{angles, quotes}
\usetikzlibrary{3d}

% Margins
\topmargin=0in
\evensidemargin=0in
\oddsidemargin=0in
\textwidth=6.5in
\textheight=9.0in
\headsep=0.25in

\AtBeginEnvironment{tcolorbox}{\small}

\newtcolorbox{imp}{enhanced,arc=0mm,colback=yorhabg,colframe=mred,leftrule=10mm,coltext=yorhafg,%
overlay={\node[anchor=west,outer sep=2pt] at (frame.west) {\includegraphics[width=6mm]{images/imageb.png}}; }}

\newtcolorbox{shortcut}{enhanced,arc=0mm,colback=yorhabg,colframe=mred,leftrule=10mm,coltext=yorhafg, coltitle=yorhabg, title=\texttt{Shortcut.}, 
overlay={\node[anchor=west,outer sep=2pt] at (frame.west) {\includegraphics[width=6mm]{images/imageb.png}}; }}

\newtcolorbox{question}[1]{
    enhanced, 
    colback=yorhabg,
    colframe=mblue,
    coltext=yorhafg,
    coltitle=yorhabg,
    attach boxed title to top left={yshift*=-\tcboxedtitleheight}, 
    title=\texttt{#1},
    boxed title size=title,
    boxed title style={%
        rounded corners=northeast, 
        rounded corners=northwest, 
        colback=tcbcolframe, 
        boxrule=0pt,
    },
    underlay boxed title={%
        \path[fill=tcbcolframe] (title.south west)--(title.south east) 
            to[out=0, in=180] ([xshift=5mm]title.east)--
            (title.center-|frame.east)
            [rounded corners=5pt] |- 
            (frame.north) -| cycle; 
    },
}

\newcommand\bb[1]{\textcolor{yorhafg}{\textbf{#1}}}

\title{\textbf{MATA31 - Assignment \#6}}
\author{Satyajit Datta \ 1012033336}
\date{\today}

\begin{document}

\maketitle

\section*{1. Textbook Questions}

\begin{question}{1.4.62}
    $f(x) = -2x^2+4, K = 0$
\end{question}

Given $I = [-2, 0]$,

\begin{align*}
    f(-2) & = -2(-2)^2 + 4 \\
    & = -2(4) + 4 \\
    & = -8 + 4 \\
    & = -4
\end{align*}

\begin{align*}
    f(0) & = -2(0)^2 + 4 \\
    & = 0 + 4 \\
    & = 4
\end{align*}

Given that $K \in [-4, 4]$ and $f(x)$ is continuous everywhere, then $\exists x, f(x) = K. \blacksquare$

\begin{question}{1.4.64}
    $f(x) = \sin(x), K = \frac{\sqrt{3}}{2}$
\end{question}

Given $I = [0, \frac{\pi}{2}]$,

\begin{align*}
    f(0) & = \sin{(0)} \\
    & = 0 \\
\end{align*}

\begin{align*}
    f(\frac{\pi}{2}) & = \sin{\frac{\pi}{2}} \\
    & = 1 \\
\end{align*}

Given that $K \in [0, 1]$ and $f(x)$ is continuous everywhere, then $\exists x, f(x) = K. \blacksquare$

\begin{question}{1.4.66}
    $f(x) = |2 - 3x|, K = 2$
\end{question}

Given $I = [1, 3]$,

\begin{align*}
    f(1) & = |2 - 3(1)| \\
    & = |2 - 3| \\
    & = | -2| \\
    & = 2
\end{align*}

\begin{align*}
    f(3) & = |2 - 3(3)| \\
    & = |2 - 9| \\
    & = | -7| \\
    & = 7
\end{align*}

Given that $K \in [2, 7]$ and $f(x)$ is continuous everywhere, then $\exists x, f(x) = K. \blacksquare$

\begin{question}{1.5.42}
    \[
        \lim_{x \to 1^+} \frac{1 - \sqrt{x}}{1-x}
    \]
\end{question}

Note that this limit exists because it is a one-sided limit from the positive direction, therefore the numerator is defined.

\begin{align*}
    &\lim_{x \to 1^+} \frac{1 - \sqrt{x}}{1-x} \\
    & =\lim_{x \to 1^+} \frac{1 - \sqrt{x}}{1-x}(\frac{1 + \sqrt{x}}{1 + \sqrt{x}}) \\
    & =\lim_{x \to 1^+} \frac{1 - x}{(1-x)(1 + \sqrt{x})} \\
    & =\lim_{x \to 1^+} \frac{1}{1 + \sqrt{x}} \\
    & = \frac{1}{1 + \sqrt{1}} \\ 
    & = \frac{1}{2}
\end{align*}


\begin{question}{1.5.60}
    \[
        \lim_{x \to 3} \frac{\sqrt{x}}{\tan^-1\sqrt{x}}
    \]
\end{question}

\begin{align*}
    &\lim_{x \to 3} \frac{\sqrt{x}}{\tan^-1\sqrt{x}} \\
    & = \frac{\sqrt{3}}{\tan^1\sqrt{3}} \\
    & = \frac{\sqrt{3}}{\frac{\pi}{3}} \\
    & = \frac{3\sqrt{3}}{\pi} \\
\end{align*}

\begin{question}{1.5.84}
    \[
        \lim_{x \to 0} \sin x \sin \frac{1}{x}
    \]
\end{question}

We will use the squeeze theorem for this.

Let $f(x) = \sin x \sin \frac{1}{x}$

Notice that:

\begin{align*}
    -1 \le \sin x \le 1 \\
    |\sin x| \le 1
\end{align*}

and, 

\begin{align*}
    -1 \le \sin \frac{1}{x }\le 1 \\
    |\sin \frac{1}{x}| \le 1 \\
    |\sin x||\sin \frac{1}{x}| \le 1 \cdot |\sin x| \\
    |\sin x||\sin \frac{1}{x}| \le |\sin x| \\
    |\sin x\sin \frac{1}{x}| \le |\sin x| \\
    -|\sin x| \le \sin x\sin \frac{1}{x} \le |\sin x|
\end{align*}

Now we have two functions, $g(x) = -|\sin x|$, and $h(x) = |\sin x|$ for which the original function lies between for the entirety of its domain.

Therefore, by squeeze theorem, if $\displaystyle \lim_{x \to 0} g(x) = \lim_{x \to 0} h(x) = L$, then the $\lim_{x \to 0} f(x) = L$

\begin{align*}
    \lim_{x \to 0} g(x) & =  \lim_{x \to 0} -|\sin x|\\
    & = -|\sin 0| \\
    & = 0 
\end{align*}

\begin{align*}
    \lim_{x \to 0} g(x) & =  \lim_{x \to 0} |\sin x|\\
    & = |\sin 0| \\
    & = 0 
\end{align*}

Therefore, by squeeze theorem, 

\[
        \lim_{x \to 0} \sin x \sin \frac{1}{x} = 0
\]

\begin{question}{1.6.34}
    Find the roots, discontinuities, and horizontal and vertical
asymptotes of 

 \[
    f(x) = \frac{1}{\tan^{-1} x} 
 \]
\end{question}

Vertical asymptotes, denominator = 0

\begin{align*}
    \tan ^{-1} x & = 0 \\
    x & = \tan 0 \\
    x = 0
\end{align*}

Therefore, there is a vertical asymptote at $x = 0$.

Since the numerator is 1, there are no roots to the function.

Vertical asymptote, $\displaystyle \lim_{x \to \infty} f(x)$ and $\displaystyle \lim_{x \to -\infty} f(x)$

\begin{align*}
    & \lim_{x \to \infty} \tan ^{-1} x \\
    & = \frac{\pi}{2} \\ & \\ 
    & \lim_{x \to \infty} \frac{1}{\tan ^{-1} x} \\
    & = \frac{1}{\frac{\pi}{2}} \\
    & = \frac{2}{\pi}
\end{align*}

\begin{align*}
    & \lim_{x \to -\infty} \tan ^{-1} x \\
    & = -\frac{\pi}{2} \\ & \\
    & \lim_{x \to \infty} \frac{1}{\tan ^{-1} x} \\
    & = \frac{1}{-\frac{\pi}{2}} \\
    & = -\frac{2}{\pi}
\end{align*}

Therefore, there are horizontal asymptotes at $y = \frac{2}{\pi}$ and $y = -\frac{2}{\pi}$


\begin{question}{1.6.46}

    \[
        \lim_{x \to -4} \frac{x + 4}{x^2+8x+16}
    \]

\end{question}


\begin{align*}
    & \lim_{x \to -4} \frac{x + 4}{x^2+8x+16} \\
    = & \lim_{x \to -4} \frac{x + 4}{(x + 4)^2} \\
    = & \lim_{x \to -4} \frac{1}{x + 4} \\
    = & \frac{1}{0} \\
    = & \text{undefined} 
\end{align*}

Therefore, since the limit is in the form $\frac{1}{0}$, it is not in an inderminate form, and the limit does not exist.

\begin{question}{1.6.70}

    \[
        \lim_{x \to 0} \frac{3\sin x + x}{x}
    \]

\end{question}

\begin{align*}
    & \lim_{x \to 0} \frac{3\sin x + x}{x} \\
    =\; & \lim_{x \to 0} \frac{3\sin x}{x} + \frac{x}{x} \\
    =\; & \lim_{x \to 0} \frac{3\sin x}{x} + 1 \\
    =\; & \lim_{x \to 0} 3\frac{\sin x}{x} + 1 \\
    =\; & 3\lim_{x \to 0} \frac{\sin x}{x} + 1 \\
    =\; & 3(1) + 1 \\
    =\; & 4
\end{align*}

\[
    \therefore \lim_{x \to 0} \frac{3\sin x + x}{x} = 4
\]

\begin{question}{1.6.78}

    \[
        \lim_{x \to 0} (1 + 2x)^{3/x}
    \]

\end{question}

\begin{center}
    Let $h = 2x$
\end{center}

\begin{align*}
    & \lim_{x \to 0} (1 + 2x)^{3/x} \\
    =\; & \lim_{h \to 0} (1 + h)^{6/h} \\
    =\; & \lim_{h \to 0} ((1 + h)^{1/h})^6 \\
    =\; & e^6 \\
\end{align*}


\section{2. Assignment Questions}

\begin{question}{B}
    Use the Intermediate Value Theorem to show that there is at least
one root of the equation $\cos\sqrt{x} = e^x - 2$ in the interval $(0, 1)$
\end{question}

Lets define the function $f(x)$ as the equation with one side, where if the function is 0, then that point is a root of the equation.

$f(x) = \cos\sqrt{x} - e^x + 2$

Define $I = [0, \frac{\pi^2}{4}]$. Therefore,

\begin{center}
    $\sqrt{x}$ is continuous on $I$. \linebreak
    $\cos{x}$ is continuous on $I$. \linebreak
    $e^x$ is continuous on $I$. \linebreak
\end{center}

Therefore, $f(x)$ is continuous on $I$, which is one of the conditions for IVT.

\begin{align*}
    f(0) & = \cos{\sqrt{0}} - e^0 + 2 \\
    & = \cos{0} - 1 + 2 \\
    & = 1 - 1 + 2 \\
    & = 2
\end{align*}

\begin{align*}
    f(1) & = \cos{\sqrt{1}} - e^1 + 2 \\
    & = \cos{1} - e + 2 \\
    & \approx 0.54 - e + 2 \\
    & = 2.54 - e
\end{align*}

Since $0 \in [2.54 - e,\; 2]$ and $f(x)$ is continuous on $I$, then there $\exists a, f(a) = 0$, which means the equation indeed does have a solution. $\blacksquare$

\begin{question}{F}
    Let f be continuous on $[a, b]$ with $f(a) < 0 < f(b)$. Then there exists
$c \in (a, b)$ such that $f(c) = 0$. To show this, proceed as follows:

    \begin{enumerate}[label=(\alph*)]
        \item Let \[A = \{x \in [a, b] : f(x) < 0\}\] \\
Show that $A$ has a supremum $c \in (a, b)$
        \item Using Problem E, and F from Assignment 5, eliminate the possibilities $f(c) < 0$ and $f(c) > 0$, concluding that $f(c) = 0$.
    \end{enumerate}
\end{question}

(a). We know that $f(x)$ is continuous on $[a, b]$.

$f(a) < 0 < f(b)$

If $A = {x \in [a, b] : f(x) < 0}$, $A = [a, 0)$
\medbreak

Let $\varepsilon > 0$ be arbitrary,
\medbreak

Suppose $\beta = \frac{0 - \varepsilon}{2}$
\medbreak

Then: $\beta > 0 - \varepsilon$ AND $\beta < 0$.
\medbreak
Therefore, $\forall \varepsilon > 0, \exists \beta \in [a, 0)\quad s.t. \quad \beta > 0 - \varepsilon$

Thus, 0 is the supremum of $A$.

Since $f$ is continuous on $[a, b]$, and $f(a) < 0 < f(b)$, then there exists $c$ such that $f(c) = 0$

Due to IVT, $\exists c \in (a, b) \quad \text{s.t} \quad c = sup(A)$.


(b). Suppose $f(c) < 0.$
 

Then, $f(c) \in A$.

Take $\beta = \frac{f(c)}{2}$.

$0 > \beta > f(c)$, therefore by contradiction, $f(c) \ne sup(A)$. Therefore $f(c) \ge 0$.

Suppose $f(c) > 0$.

Choose $\beta = \frac{f(c)}{2}$. Note $\beta > 0$.

Note that $0 < \beta < f(c)$, meaning there exists a smaller value than $f(c)$ which is not in A, making $f(c)$ not the supremum.

Therefore, by contradiction, $f(c) \le 0$.

Since $f(c) \ge 0$ and $f(c) \le 0$, then $f(c) = 0$.


\begin{question}{G}
    Use the Intermediate Value Theorem to prove that, at any given
time, there exist two antipodal points on the Earth’s surface (i.e., exactly opposite each other) with exactly the same temperature. (Hint:
consider defining a function that takes the temperature at any point
and subtracts the temperature at its antipodal point.)
\end{question}

Let $f(x) = t(x) - a(x)$, where t(x) is the temperature at x, and a(x) is the temperature at the antipodal point of x.

Suppose $t(a) - a(a) = \beta$. Then $a(a) - t(a) = -\beta$. Since a(a) is the antipodal point of t(a), then t(a) is also the antipodal of a(a), therefore when going from one side to the other, you cross the range of values from $[-\beta, \beta]$. Since $0 \in [-\beta, \beta]$, and f(x) is continuous everywhere, then by IVT, there exists a point where the difference between two antipodal is 0, aka the same temperature
\end{document}