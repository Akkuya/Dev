\documentclass[]{article}

% Imports the catppuccin theme, using the mocha flavor,
% from the directory above. Actual implementation
% wouldn't need the import package unless the theme
% and the document are in different directories.
\usepackage{import}
\usepackage{xcolor}
\usepackage{fancyhdr}
\usepackage{cancel}
\usepackage{mathtools}

% For permutations and combinations
\newcommand\Myperm[2][^n]{\prescript{#1\mkern-2.5mu}{}P_{#2}}
\newcommand\Mycomb[2]{\prescript{#1\mkern-0.5mu}{}C_{#2}}

% Colors
\definecolor{yorhabg}{HTML}{FFFFFF}
\definecolor{yorhafg}{HTML}{000000}
\definecolor{yorhagrid}{HTML}{B5AF9C}
\definecolor{mred}{HTML}{D67069}
\definecolor{mblue}{HTML}{6887A1}

\pagecolor{yorhabg}
\color{yorhafg}

\usepackage{preamble}

% Removes padding above title
\usepackage{titling}
\setlength{\droptitle}{-10em}

% Font package
\usepackage[T1]{fontenc}

\usepackage{fouriernc}

\usepackage{sectsty}
\usepackage{graphicx}
\usepackage{amsmath}
\usepackage{amsfonts}
\usepackage{amssymb}
\usepackage[skins, most]{tcolorbox}

\DeclareMathOperator{\sgn}{sgn}

\usepackage{tikz}
\usepackage{eso-pic}
\usetikzlibrary{calc,shadows.blur}
\usetikzlibrary{angles, quotes}
\usetikzlibrary{3d}

% Margins
\topmargin=0in
\evensidemargin=0in
\oddsidemargin=0in
\textwidth=6.5in
\textheight=9.0in
\headsep=0.25in

\AtBeginEnvironment{tcolorbox}{\small}

\newtcolorbox{imp}{enhanced,arc=0mm,colback=yorhabg,colframe=mred,leftrule=10mm,coltext=yorhafg,%
overlay={\node[anchor=west,outer sep=2pt] at (frame.west) {\includegraphics[width=6mm]{images/imageb.png}}; }}

\newtcolorbox{shortcut}{enhanced,arc=0mm,colback=yorhabg,colframe=mred,leftrule=10mm,coltext=yorhafg, coltitle=yorhabg, title=\texttt{Shortcut.}, 
overlay={\node[anchor=west,outer sep=2pt] at (frame.west) {\includegraphics[width=6mm]{images/imageb.png}}; }}

\newtcolorbox{question}[1]{
    enhanced, 
    colback=yorhabg,
    colframe=mblue,
    coltext=yorhafg,
    coltitle=yorhabg,
    attach boxed title to top left={yshift*=-\tcboxedtitleheight}, 
    title=\texttt{#1},
    boxed title size=title,
    boxed title style={%
        rounded corners=northeast, 
        rounded corners=northwest, 
        colback=tcbcolframe, 
        boxrule=0pt,
    },
    underlay boxed title={%
        \path[fill=tcbcolframe] (title.south west)--(title.south east) 
            to[out=0, in=180] ([xshift=5mm]title.east)--
            (title.center-|frame.east)
            [rounded corners=5pt] |- 
            (frame.north) -| cycle; 
    },
}
\pagestyle{fancy}
\fancyhf{}  % Clear all default header/footer
\fancyhead[L]{\small Satyajit Datta \\ 1012033336}  % Top-left, small font
\newcommand\bb[1]{\textcolor{yorhafg}{\textbf{#1}}}

\title{\textbf{MATA31 - Assignment \#3}}
\author{ Satyajit Datta \\ 1012033336}
\date{\today}

\begin{document}

\maketitle

\section*{Textbook Questions}
 
\begin{question}{Question 26.}
Determine whether the statement: There exist real numbers $x$ and $y$ such that $x + y = 4$ is true Justify your answer.
\end{question}
\begin{center}
    Since this statement uses the phrase "there exists", only one solution is necessary to make this statement true. Let $x = 1, y=3$, then: 
\end{center}
\begin{align*}
    & x+y = 4\\
    \Rightarrow & 1 + 3 = 4 \\
    \Rightarrow & 4 = 4
\end{align*}
\begin{center}
    $\therefore$ The statement is true. $\blacksquare$
\end{center}

\begin{question}{Question 50.}
Suppose $A$ and $B$ represent logical statements, write the converse and contrapositive of: $\neg A \Rightarrow \neg B$. 
\end{question}
\begin{align*}
    & \neg A \Rightarrow \neg B \\
    \text{Converse: } & \neg B \Rightarrow \neg A \\
    \text{Contrapositive: } & B \Rightarrow A
\end{align*}

\begin{question}{Question 62.}
Write the converse and contrapositive of: If $x < -2$, then $|x| = -x$. Provide counterexamples if the original, the
converse, and/or the contrapositive statements are false.
\end{question}
\begin{align*}
    & \text{Original: If } x < -2, \text{ then } |x| = -x. \\
    & \text{Converse: If } |x| = -x, \text{ then } x < -2. \\
    & \text{Contrapositive: If } |x| \neq -x, \text{ then } x \ge -2.
\end{align*}
\begin{center}
    The original statement is true, since if $x < -2$, then $x$ is negative, and the absolute value of a negative number is its negation. \\
    The converse is false, since if $x = -1$, then $|x| = -x$ but $x \not< -2$. \\
    The contrapositive is true, since if $|x| \neq -x$, then $x$ is positive or zero, which means $x \ge -2$.
\end{center}

\begin{question}{Question 64.}
Write the converse and contrapositive of: If $x$ is positive and rational, then $x - 1$ is positive and rational.
\end{question}
\begin{align*}
    & \text{Original: If } x \text{ is positive and rational, then } x - 1 \text{ is positive and rational.} \\
    & \text{Converse: If } x - 1 \text{ is positive and rational, then } x \text{ is positive and rational.} \\
    & \text{Contrapositive: If } x - 1 \text{ is not positive or not rational, then } x \text{ is not positive or not rational.}
\end{align*}
\begin{center}
    The original statement is true, since if $x$ is positive and rational, then $x - 1$ is also positive and rational. \\
    The converse is false, since if $x - 1 = 0$, then $x = 1$, which is positive and rational, but if $x - 1 = -1$, then $x = 0$, which is not positive. \\
    The contrapositive is true, since if $x - 1$ is not positive, then $x \le 1$, and if $x - 1$ is not rational, then $x$ is not rational.
\end{center}

\begin{question}{Question 86.}
Using the definition of absolute value and systems of inequalities, prove that for any real numbers $x$ and $c$, and for any positive real number $\delta$, the statement: $|x-c| < \delta \iff x \in (c-\delta, c+\delta)$
\end{question}

\[
\text{We must prove that } |x-c| < \delta \iff x \in (c-\delta, c+\delta),
\]
For all real numbers $x, c$, and all positive real numbers $\delta$.

\underline{\textbf{Proof.}}  

Let $\delta > 0$ be arbitrary.  

\medskip

\noindent \textbf{($\Rightarrow$)}  
Assume $|x-c| < \delta$.  
\begin{align*}
    |x-c| < \delta 
    &\;\;\equiv\;\; -\delta < x-c < \delta 
    && \text{(by properties of absolute values)} \\[6pt]
    &\;\;\equiv\;\; c-\delta < x < c+\delta 
    && \text{(by algebra)} 
\end{align*}
Thus $x \in (c-\delta, c+\delta)$.  

\medskip

\noindent \textbf{($\Leftarrow$)}  
Assume $x \in (c-\delta, c+\delta)$.  
\begin{align*}
    x \in (c-\delta, c+\delta) 
    &\;\;\equiv\;\; c-\delta < x < c+\delta 
    && \text{(by definition of open interval)} \\[6pt]
    &\;\;\equiv\;\; -\delta < x-c < \delta 
    && \text{(by algebra)} \\[6pt]
    &\;\;\equiv\;\; |x-c| < \delta 
    && \text{(by properties of absolute values)}
\end{align*}
Thus $|x-c| < \delta$.  

\medskip

Since both directions hold, we conclude that
\[
|x-c| < \delta \iff x \in (c-\delta, c+\delta).
\]

As required to show. $\blacksquare$

\vspace{0.1in}
\hrule
\vspace{0.1in}

\section*{2. Non-textbook Questions}
\begin{question}{1.A}
Prove that $\log_k(ab) = \log_k(a) + \log_k(b), k > 0, k \ne 1$ and $a, b$ are any two positive numbers.
\end{question}
\[
\text{We must prove that } \log_k(ab) = \log_k(a) + \log_k(b)
\]
For any real positive numbers $a, b$, and any $k$, such that $k>0, k\ne1$.

\underline{\textbf{Proof.}}  

Let \(a>0, b>0\) and \(k>0, k\ne1\) be arbitrary. 

\medskip

Let $x = \log_k(a)$, and let $y = \log_k(b)$

Then by definitions of logarithms: 

\[
k^x = a \quad \text{and} \quad k^y = b.
\]

Therefore,
\begin{align*}
    ab &= k^x \cdot k^y && \text{(Definition of $x = \log_k(a)$ and $y = \log_k(b)$)}\\
       &= k^{x+y} && \text{(By properties of exponents.)}\\
    \log_k(ab) &= \log_k(k^{x+y}) && \text{(By algebra)}\\
               &= x+y && \text{(By properties of logarithms)}\\
               &= \log_k(a) + \log_k(b) && \text{(Substitute $x = \log_k(a)$, $y = \log_k(b)$)}
\end{align*}

As required to show. $\blacksquare$.

\begin{question}{A.2}
    Provide a counter-example to show that the relation
    \begin{center}
        $\log_k{(a+b)} = \log_k(a) + \log_k(b)$
    \end{center}
    is not true.
\end{question}

\underline{\textbf{Proof.}}  

Choose $a = 625, b = 25, k = 5$.

Then, 
\begin{align*}
    \log_k(a+b) &= \log_5(625 + 25) &&= \log_5(650) \approx 4.02437,\\
    \log_k(a) + \log_k(b) &= \log_5(625) + \log_5(25) &&= 4 + 2 = 6.
\end{align*}

Therefore, the relation
\[
\log_k(a+b) = \log_k(a) + \log_k(b)
\]
is not true, as required to show. $\blacksquare$

\begin{question}{2}
    Given that $0 \le a \le b$, show that
    \begin{center}
        $a \le \sqrt{ab} \le \frac{a+b}{2} \le b$
    \end{center}
\end{question}

\underline{\textbf{Proof.}}  

Let $0 \le a \le b$

Then: 
\begin{align*}
    a \le b
    &\;\;\equiv\;\; a^2 \le ab 
    && \text{(by algebra)} \\[6pt]
    &\;\;\equiv\;\; \sqrt{a^2} \le \sqrt{ab}
    && \text{(by algebra)} \\[6pt]
    &\;\;\equiv\;\; a \le \sqrt{ab} 
    && \text{(by algebra)}
\end{align*}

Note that since $\sqrt{x}$ is a positive increasing function and $a, b \ge 0$, then $\sqrt{ab} \ge 0$, meaning the inequality is preserved.

Next:
\begin{align*}
    a \le b
    &\;\;\equiv\;\; b - a \ge 0
    && \text{(by algebra)} \\[6pt]
    &\;\;\equiv\;\; (b-a)^2 \ge 0
    && \text{(squaring both sides)} \\[6pt]
    &\;\;\equiv\;\; b^2 - 2ab + a^2 \ge 0 
    && \text{(binomial expansion)} \\[6pt]
    &\;\;\equiv\;\; b^2 + a^2 \ge 2ab 
    && \text{(by algebra)} \\[6pt]
    &\;\;\equiv\;\; b^2 + 2ab + a^2 \ge 4ab 
    && \text{(adding 2ab to both sides)} \\[6pt]
    &\;\;\equiv\;\; (a + b)^2 \ge 4ab 
    && \text{(factoring)} \\[6pt]
    &\;\;\equiv\;\; \frac{(a + b)^2}{4} \ge ab 
    && \text{(by algebra)} \\[6pt]
    &\;\;\equiv\;\; \sqrt{\frac{(a + b)^2}{4}} \ge \sqrt{ab}
    && \text{(square root of both sides)} \\[6pt]
    &\;\;\equiv\;\; \frac{a+b}{2} \ge \sqrt{ab}
    && \text{(by algebra)} \\[6pt]
    &\;\;\equiv\;\; \sqrt{ab} \le \frac{(a+b)}{2}
\end{align*}

Note that since $\sqrt{x}$ is a positive increasing function and $a, b \ge 0$, then the square root is nonnegative, meaning the inequality is preserved.

\medskip

Finally:
\begin{align*}
    a\le  b
    &\;\;\equiv\;\; a + b \le b + b
    && \text{(by algebra)} \\[6pt]
    &\;\;\equiv\;\; a + b \le 2b
    && \text{(by algebra)} \\[6pt]
    &\;\;\equiv\;\; \frac{a+b}{2} \le b
    && \text{(by algebra)} \\[6pt]
\end{align*}

Given that $a \le \sqrt{ab}$, $\sqrt{ab} \le \frac{(a+b)}{2}$, and $\frac{a+b}{2} \le b$: By properties of inequalities we can state that:

\begin{center}
    $a \le \sqrt{ab} \le \frac{a+b}{2} \le b$
\end{center}

As required to prove. $\blacksquare$.

\begin{question}{3.A}
    Prove or disprove (by counterexample):
        \begin{center}
            Let $x, y \in \mathbb{R}$. If $x, y \in \mathbb{Q}$, then $x+y \in \mathbb{Q}$
        \end{center}
\end{question}

We will prove this statement.

\medbreak

\textbf{Proof.}  

Let \(x, y \in \mathbb{Q}\).  

Then, by definition of rational numbers, there exist integers \(a, b, c, d \in \mathbb{Z}\) with \(b \neq 0\) and \(d \neq 0\) such that  
\[
x = \frac{a}{b} \quad \text{and} \quad y = \frac{c}{d}.
\]

Consider the sum:
\begin{align*} 
    x + y 
    \;\;\equiv\;\; \frac{a}{b} + \frac{c}{d} 
    && \text{(substitute x and y)} \\[6pt] 
    \;\;\equiv\;\; \frac{ad}{bd} + \frac{bc}{bd} 
    && \text{(algebra)} \\[6pt] 
    \;\;\equiv\;\; \frac{ad + bc}{bd} 
    && \text{(algebra)} \\[6pt] 
\end{align*}

Since \(a, b, c, d \in \mathbb{Z}\), both the numerator \(ad + bc\) and the denominator \(bd\) are integers, and \(bd \neq 0\).  

Thus, \(x+y\) is a ratio of two integers, which means \(x+y \in \mathbb{Q}\). \(\blacksquare\)

\begin{question}{3.B}
    Prove or disprove (by counterexample):
        \begin{center}
            Let $x, y \in \mathbb{R}$. If $x \notin \mathbb{Q}$ and $y \notin \mathbb{Q}$, then $x+y \notin \mathbb{Q}$
        \end{center}
\end{question}

\underline{\textbf{Proof (by counterexample).}}  

We will disprove the statement by providing a counterexample.  

Choose 
\[
x = \sqrt{2} \quad \text{and} \quad y = -\sqrt{2}.
\]

Then, 
\[
x + y = \sqrt{2} + (-\sqrt{2}) = 0.
\]

Since \(0 \in \mathbb{Q}\), the sum \(x+y\) is rational.  

Therefore, the statement "If \(x \notin \mathbb{Q}\) and \(y \notin \mathbb{Q}\), then \(x+y \notin \mathbb{Q}\)" is not true. \(\blacksquare\)

\begin{question}{3.C}
    Prove or disprove (by counterexample):
        \begin{center}
            Let $x, y \in \mathbb{R}$. If $x \in \mathbb{Q}$ and $y \notin \mathbb{Q}$, then $xy \notin \mathbb{Q}$
        \end{center}
\end{question}

\underline{\textbf{Proof (by counterexample).}}  

We will disprove the statement by providing a counterexample.  

Choose 
\[
x = 0 \quad \text{and} \quad y = \sqrt{2}.
\]

Then, 
\[
xy = (0)(\sqrt{2}) = 0.
\]

Since \(0 \in \mathbb{Q}\), the product \(xy\) is rational.  

Therefore, the statement "If \(x \in \mathbb{Q}\) and \(y \notin \mathbb{Q}\), then \(xy \notin \mathbb{Q}\)" is not true. \(\blacksquare\)

\begin{question}{3.D}
    Prove or disprove (by counterexample):
        \begin{center}
            Let $x, y \in \mathbb{R}$. If $x \in \mathbb{Q}$ and $y \notin \mathbb{Q}$, then $x + y \notin \mathbb{Q}$
        \end{center}
\end{question}

\underline{\textbf{Proof (by contradiction).}}  

Let \(x \in \mathbb{Q}\) and \(y \notin \mathbb{Q}\) be arbitrary.  

For the sake of contradiction, assume
\[
x + y \in \mathbb{Q}.
\]

By definition of rational numbers, there exist integers \(a, b, c, d \in \mathbb{Z}\) with \(b \neq 0\) and \(d \neq 0\) such that
\[
x = \frac{a}{b} \quad \text{and} \quad x + y = \frac{c}{d}.
\]

Then, solving for \(y\), we have:
\begin{align*}
y &= (x+y) - x && \text{(algebra)} \\[1mm]
  &= \frac{c}{d} - \frac{a}{b} && \text{(substitute \(x\) and \(x+y\))} \\[1mm]
  &= \frac{bc - ad}{bd} && \text{(common denominator)}
\end{align*}

Since \(a, b, c, d \in \mathbb{Z}\), both the numerator \(bc - ad\) and the denominator \(bd\) are integers, and \(bd \neq 0\).  

Hence, \(y = \frac{bc - ad}{bd} \in \mathbb{Q}\), which contradicts our assumption that \(y \notin \mathbb{Q}\).  

Therefore, our assumption is false, and the statement  
\[
\text{If } x \in \mathbb{Q} \text{ and } y \notin \mathbb{Q}, \text{ then } x+y \notin \mathbb{Q}
\] 

is true. \(\blacksquare\)

\end{document}