\documentclass[]{article}

% Imports the catppuccin theme, using the mocha flavor,
% from the directory above. Actual implementation
% wouldn't need the import package unless the theme
% and the document are in different directories.
\usepackage{import}
\usepackage{xcolor}
% \usepackage{fancyhdr}
\usepackage{cancel}
\usepackage{mathtools}

% For permutations and combinations
\newcommand\Myperm[2][^n]{\prescript{#1\mkern-2.5mu}{}P_{#2}}
\newcommand\Mycomb[2]{\prescript{#1\mkern-0.5mu}{}C_{#2}}

% Colors
\definecolor{yorhabg}{HTML}{FFFFFF}
\definecolor{yorhafg}{HTML}{000000}
\definecolor{yorhagrid}{HTML}{B5AF9C}
\definecolor{mred}{HTML}{D67069}
\definecolor{mblue}{HTML}{6887A1}

\pagecolor{yorhabg}
\color{yorhafg}

\usepackage{preamble}

% Removes padding above title
\usepackage{titling}
\setlength{\droptitle}{-10em}

% Font package
\usepackage[T1]{fontenc}

\usepackage{fouriernc}

\usepackage{sectsty}
\usepackage{graphicx}
\usepackage{amsmath}
\usepackage{amsfonts}
\usepackage{amssymb}
\usepackage[skins, most]{tcolorbox}
\usepackage{enumitem}

\DeclareMathOperator{\sgn}{sgn}

\usepackage{tikz}
\usepackage{eso-pic}
\usetikzlibrary{calc,shadows.blur}
\usetikzlibrary{angles, quotes}
\usetikzlibrary{3d}

% Margins
\topmargin=0in
\evensidemargin=0in
\oddsidemargin=0in
\textwidth=6.5in
\textheight=9.0in
\headsep=0.25in

\AtBeginEnvironment{tcolorbox}{\small}

\newtcolorbox{imp}{enhanced,arc=0mm,colback=yorhabg,colframe=mred,leftrule=10mm,coltext=yorhafg,%
overlay={\node[anchor=west,outer sep=2pt] at (frame.west) {\includegraphics[width=6mm]{images/imageb.png}}; }}

\newtcolorbox{shortcut}{enhanced,arc=0mm,colback=yorhabg,colframe=mred,leftrule=10mm,coltext=yorhafg, coltitle=yorhabg, title=\texttt{Shortcut.}, 
overlay={\node[anchor=west,outer sep=2pt] at (frame.west) {\includegraphics[width=6mm]{images/imageb.png}}; }}

\newtcolorbox{question}[1]{
    enhanced, 
    colback=yorhabg,
    colframe=mblue,
    coltext=yorhafg,
    coltitle=yorhabg,
    attach boxed title to top left={yshift*=-\tcboxedtitleheight}, 
    title=\texttt{#1},
    boxed title size=title,
    boxed title style={%
        rounded corners=northeast, 
        rounded corners=northwest, 
        colback=tcbcolframe, 
        boxrule=0pt,
    },
    underlay boxed title={%
        \path[fill=tcbcolframe] (title.south west)--(title.south east) 
            to[out=0, in=180] ([xshift=5mm]title.east)--
            (title.center-|frame.east)
            [rounded corners=5pt] |- 
            (frame.north) -| cycle; 
    },
}

\newcommand\bb[1]{\textcolor{yorhafg}{\textbf{#1}}}

\title{\textbf{CSCA67 - Exercises \#9}}
\author{Satyajit Datta \ 1012033336}
\date{\today}

\begin{document}

\maketitle
\begin{question}{2.1}
    The famous Fibonacci sequence is defined as follows.
    \begin{align*}
        &F_0 = 0 \\
        &F_1 = 1 \\
        &F_n = F_{n-1} + F_{n-2}
    \end{align*}
    Perhaps surprisingly, the $n^{th}$ Fibonacci number can be calculated for any $n$ by using the following
formula:
    \[
        F_n = \frac{\phi^n - \psi^n}{\sqrt{5}} \text{ where } \phi = \frac{1 + \sqrt{5}}{2} \text{ and } \psi = \frac{1 - \sqrt{5}}{2}
    \]
Prove this formula.
\end{question}

Let $P(n) \text{ be } F_n = \frac{\phi^n - \psi^n}{\sqrt{5}}$
\medbreak
Note the following:

$F_0 = 0$

$F_1 = 1$

$\phi = \frac{1 + \sqrt{5}}{2}$

$\psi = \frac{1 - \sqrt{5}}{2}$

WTS: $\forall k\ge0, (\forall 0\le i < k, (P(i)) \rightarrow P(k) )$

\newlength{\eqindent}
\settowidth{\eqindent}{$\frac{\phi^0 - \psi^0}{\sqrt{5}}$}
\newlength{\eqaindent}
\settowidth{\eqaindent}{$F_{k-1} + F_{k-2}$}

\underline{\textbf{Proof.}}

\begin{flalign}
    &\text{Let k be arbitrary.} \\
    &\quad\text{Suppose } k \ge 0 \\
    &\quad\quad\text{Suppose} \forall 0 \le i < k, P(i) \\
    &\quad\quad\quad\text{Suppose } k = 0 \\
    &\quad\quad\quad\quad \frac{\phi^0 - \psi^0}{\sqrt{5}} = \frac{1 - 1}{\sqrt{5}} & \text{Math} \\
    &\quad\quad\quad\quad\quad \hspace*{\eqindent} = \frac{0}{\sqrt{5}} \\
    &\quad\quad\quad\quad\quad \hspace*{\eqindent} = 0 \\
    &\quad\quad\quad\quad \frac{\phi^0 - \psi^0}{\sqrt{5}} = F_0 \\
    &\quad\quad\quad\quad P(0) \\
    &\quad\quad\quad\text{Suppose } k = 1 \\
    &\quad\quad\quad\quad \frac{\phi - \psi}{\sqrt{5}} = \frac{\frac{2\sqrt{5}}{2}}{\sqrt{5}} & \text{Math} \\
    &\quad\quad\quad\quad\quad \hspace*{\eqindent} = \frac{2\sqrt{5}}{2\sqrt{5}} \\
    &\quad\quad\quad\quad\quad \hspace*{\eqindent} = 1 \\
    &\quad\quad\quad\quad \frac{\phi - \psi}{\sqrt{5}} = F_1 \\
    &\quad\quad\quad\quad P(1) \\
    &\quad\quad\quad\text{Suppose } k \ge 2 \\
    &\quad\quad\quad\quad F_k = F_{k-1} + F_{k-2} \\
    &\quad\quad\quad\quad\quad\hspace*{\eqaindent} = \frac{\phi^{k-1} - \psi^{k-1} + \phi^{k-2} - \psi^{k-2}}{\sqrt{5}}\\
    % &\quad\quad\quad\quad\quad\hspace*{\eqaindent} = \frac{\phi^{k-2}\(\phi + 1\) - \psi^{k-2}\(\psi + 1\)}{\sqrt{5}}\\
    % &\quad\quad\quad\quad\quad\hspace*{\eqaindent} = \frac{\phi^{k-2}\(\phi^2\) - \psi^{k-2}\(\psi^2\)}{\sqrt{5}}\\
    % &\quad\quad\quad\quad\quad\hspace*{\eqaindent} = \frac{\phi^{k} - \psi^{n}}{\sqrt{5}}\\
    &\quad\quad\quad\quad F_n = \frac{\phi^{k} - \psi^{k}}{\sqrt{5}}\\
    &\quad\quad\quad\quad F_n = \frac{\phi^{k} - \psi^{k}}{\sqrt{5}}\\
\end{flalign}

\end{document}